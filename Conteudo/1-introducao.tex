\section{INTRODUÇÃO}
    A crescente adoção de redes Wi-Fi em instituições de ensino superior e técnica amplia as possibilidades de comunicação, mas também expande a superfície de ataque para práticas de engenharia social. Ataques como \emph{phishing} e \emph{evil twin} — pontos de acesso Wi-Fi fraudulentos — podem induzir usuários desavisados a fornecer credenciais sensíveis em portais falsos e viabilizar interceptações do tipo \emph{Man-in-the-Middle} \cite{james2020}.
    
    No contexto acadêmico, a ausência de mecanismos robustos de autenticação e a familiaridade limitada dos usuários com boas práticas de segurança tornam esses ambientes particularmente vulneráveis. Estudos demonstram que redes abertas em campi universitários permitem a aplicação prática de técnicas como \emph{MAC spoofing}, \emph{SSL stripping} e portais cativos falsos com alto grau de efetividade \cite{yang2012}.
    
    Além disso, campanhas reais de \emph{phishing} por QR codes — chamadas de \emph{quishing} — revelam que usuários frequentemente fornecem credenciais quando são expostos a fluxos de autenticação visualmente confiáveis, mesmo em ambientes controlados como universidades \cite{sharevski2022}.
    
    Este estudo foi conduzido durante o Seminário Tópicos II e estruturado em três frentes de atuação, organizadas por equipes distintas. A Equipe 1 foi responsável pela infraestrutura da rede Wi‑Fi, criando um ponto de acesso aberto que, ao ser acessado, redirecionava automaticamente os usuários para uma página falsa do SUAP com o intuito de capturar dados inseridos pelos participantes. 
    
    A Equipe 2 desenvolveu e apresentou o portal falso durante o minicurso, utilizando estratégias de engenharia social — explorando a confiança nos apresentadores — para elevar a taxa de interação, uma prática reconhecida como eficaz em campanhas de conscientização digital \cite{geisler2024}. Por sua vez, a Equipe 3 ficou encarregada de processar os dados coletados, transformando-os em informações relevantes e apresentando análises críticas com foco na conscientização sobre os riscos de segurança digital.
    
    A proposta, além de prática, teve caráter educativo, promovendo reflexões sobre vulnerabilidades comuns em ambientes conectados e estimulando o desenvolvimento do pensamento crítico em relação à segurança da informação.