\section{INTRODUÇÃO}
    A crescente adoção de redes Wi-Fi em instituições de ensino superior e técnica amplia as possibilidades de comunicação, mas também expande a superfície de ataque para práticas de engenharia social. Ataques como \emph{phishing} e \emph{evil twin} — pontos de acesso Wi-Fi fraudulentos — podem induzir usuários desavisados a fornecer credenciais sensíveis em portais falsos e viabilizar interceptações do tipo \emph{Man-in-the-Middle} \cite{james2020}.
    
    No contexto acadêmico, a ausência de mecanismos robustos de autenticação e a familiaridade limitada dos usuários com boas práticas de segurança tornam esses ambientes particularmente vulneráveis. Estudos demonstram que redes abertas em campi universitários permitem a aplicação prática de técnicas como \emph{MAC spoofing}, \emph{SSL stripping} e portais cativos falsos com alto grau de efetividade \cite{yang2012}.
    
    Além disso, campanhas reais de \emph{phishing} por QR codes — chamadas de \emph{quishing} — revelam que usuários frequentemente fornecem credenciais quando são expostos a fluxos de autenticação visualmente confiáveis, mesmo em ambientes controlados como universidades \cite{sharevski2022}.
    
    Neste relato de experiência, apresentamos uma atividade prática que desenvolvemos durante o Seminário Tópicos II, motivada pela necessidade de expor, de forma controlada, as fragilidades de segurança em redes abertas. Estruturamos nossa atuação em três frentes complementares, organizadas em equipes. Nossa Equipe 1 assumiu a responsabilidade pela infraestrutura da rede Wi‑Fi, configurando um ponto de acesso aberto que redirecionava os usuários automaticamente para uma página falsa do SUAP, com o objetivo de capturar dados inseridos. 
    
    Já a Equipe 2 dedicou-se ao desenvolvimento e apresentação do portal falso durante o minicurso, aplicando estratégias de engenharia social — e explorando a confiança natural nos apresentadores — para aumentar a taxa de interação, uma tática que percebemos ser altamente eficaz em simulações de conscientização \cite{geisler2024}. Por fim, a Equipe 3 ficou encarregada de processar os dados que coletamos, transformando-os em informações visuais e análises críticas para embasar nossa discussão sobre segurança digital.
    
    Acreditamos que essa proposta, além de técnica, possui um forte caráter educativo, e buscamos com ela promover uma reflexão genuína sobre as vulnerabilidades que vivenciamos diariamente em ambientes conectados.