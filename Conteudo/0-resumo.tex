\begin{spacing}{1.0}
{\fontsize{12}{14}\selectfont 
\noindent \textbf{Resumo:} Este relato de experiência, nascido das atividades da disciplina de Cibersegurança do Instituto Federal de Rondônia (IFRO) – Campus Ji-Paraná, descreve a vivência de um experimento acadêmico de captura \emph{ética} de informações em redes Wi-Fi abertas realizado durante a Semana de Extensão. Nosso objetivo principal foi despertar a conscientização sobre segurança digital, demonstrando na prática os riscos do uso descuidado de redes públicas. A implantação de um portal falso em ambiente controlado nos permitiu registrar credenciais, analisar o comportamento dos usuários e, a partir dessa observação direta, propor estratégias educativas mais eficazes para a comunidade acadêmica.
}
\end{spacing}


\vspace{0.3cm}
\noindent \textbf{Palavras Chave:} Cibersegurança. Wi-Fi Público. Phishing. Engenharia Social. M5StickC.