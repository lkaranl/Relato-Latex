\begin{spacing}{1.0}
{\fontsize{12}{14}\selectfont 
\noindent \textbf{Resumo:} Este trabalho, idealizado nas aulas de Cibersegurança do Instituto Federal de Rondônia (IFRO) – Campus Ji-Paraná, durante a Semana de Extensão, descreve um experimento acadêmico de captura \emph{ética} de informações de usuários em redes Wi-Fi abertas. O objetivo foi conscientizar sobre boas práticas de segurança, tais como adoção de senhas fortes e exclusivas, verificação de certificados HTTPS, uso de autenticação multifator, atualização de software e cautela ao utilizar redes públicas. A implantação de portais falsos em ambiente controlado possibilitou registrar credenciais e parâmetros de acesso, analisar vetores de exposição e propor estratégias educativas que reforcem a confidencialidade, integridade e disponibilidade dos dados.
}
\end{spacing}


\vspace{0.3cm}
\noindent \textbf{Palavras Chave:} Cibersegurança. Wi-Fi Público. Phishing. Engenharia Social. M5StickC.