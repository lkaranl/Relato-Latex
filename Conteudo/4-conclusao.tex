\section{CONSIDERAÇÕES FINAIS OU CONCLUSÃO}
    Nossa experiência evidenciou a suscetibilidade dos usuários a ataques de phishing bem elaborados. A diversidade de dispositivos e navegadores que interceptamos nos mostrou que não existe uma "plataforma segura" por padrão; todos os ambientes demonstraram vulnerabilidade quando o fator humano é explorado. A consistência dos acessos ao longo do tempo validou nossa abordagem e reforçou a seriedade do problema.
    
    Concluímos, com esta vivência, que a simples conscientização teórica não é suficiente. Sites falsos em redes abertas representam um risco real e imediato. Com base no que aprendemos, recomendamos fortemente:
    \begin{enumerate}
      \item A realização periódica de treinamentos práticos que simulem situações reais de phishing, tanto em computadores quanto em celulares.
      \item A revisão institucional das políticas de segurança para redes sem fio.
      \item O estudo de viabilidade para implementar sistemas que detectem esse tipo de anomalia na rede.
    \end{enumerate}
    
    Essa atividade foi fundamental para nosso desenvolvimento acadêmico, permitindo-nos ver na prática o que a teoria prevê sobre Engenharia Social.