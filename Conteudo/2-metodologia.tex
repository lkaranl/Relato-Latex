\section{MATERIAIS E MÉTODOS / PROCEDIMENTOS METODOLÓGICOS}
    A infraestrutura do experimento foi baseada no microcontrolador \emph{M5StickC}, um dispositivo compacto e de baixo consumo energético. Ele é equipado com o SoC \emph{ESP32-PICO-D4}, que apresenta arquitetura dual-core com frequência de 240\,MHz, 
    520 KB de SRAM, conectividade Wi‑Fi (802.11 b/g/n) e 4 MB de memória \emph{flash} \cite{esp32_pico_datasheet}.
        
    O dispositivo possui uma tela TFT de 0{,}96$''$ com resolução de 80×160 pixels (driver ST7735SV), e é alimentado por uma bateria interna Li‑Po de 95 mAh a 3{,}7 V, gerenciada pelo circuito PMIC \emph{AXP192}, responsável também por fornecer energia aos demais componentes internos \cite{m5stickc_doc}.
    
    O \emph{M5StickC} integra diversos sensores e periféricos: IMU \emph{MPU6886}, microfone digital \emph{SPM1423}, LED vermelho, emissor infravermelho (IR), relógio de tempo real \emph{BM8563}, buzzer, dois botões físicos e antena de 2{,}4 GHz \cite{m5stickc_doc}. Para comunicação e expansão, disponibiliza uma porta Grove e os pinos G0, G26 e G36. Suas dimensões são de aproximadamente 48{,}2 × 25{,}5 × 13{,}7 mm, com peso de cerca de 15{,}1 g e faixa de temperatura de operação entre 0 °C e 60 °C \cite{digi_m5stickc}.

    \subsection{Implementação do WIFI}
    Aqui descrevemos os procedimentos adotados para a criação e operação do ponto de acesso Wi-Fi falso utilizado no experimento, abordando desde a configuração do dispositivo até os desafios técnicos enfrentados durante sua execução.
    
    \subsubsection{Configuração do Ambiente}
       O \emph{M5StickC} foi configurado para atuar como ponto de acesso (\emph{access point}) com firmware personalizado baseado na plataforma ESP32-Arduino \cite{m5stickc_doc}. Embora o dispositivo ofereça um \emph{captive portal} nativo, foi necessário modificar o firmware para redirecionar todas as requisições HTTP para um servidor local embarcado, responsável por exibir uma réplica fiel da tela de login do Facebook (ver Figura~\ref{fig:loginfacebook}). Essa página, visualmente idêntica à original, foi projetada para capturar as credenciais inseridas pelos usuários, incluindo endereço de e-mail e senha.

   \begin{figure}[ht]
      \centering
      \includegraphics[width=0.8\linewidth]{img/login_facebook.png}
      \caption{Tela falsa de login do Facebook utilizada no experimento.}
      \label{fig:loginfacebook}
    \end{figure}

    Durante os testes, observou-se que a antena integrada do dispositivo possui alcance limitado (cerca de 10\,m). Tentativas de expandir esse alcance com o uso de roteadores como repetidores foram descartadas, uma vez que, embora os dispositivos conseguissem conectar-se à rede, o redirecionamento para o portal falso não era executado corretamente. Dessa forma, o \emph{M5StickC} foi posicionado manualmente em pontos estratégicos de circulação no campus, maximizando o alcance efetivo da rede e a chance de coleta de dados.
    
    A autonomia energética também se mostrou um desafio. A bateria interna de 95\,mAh permitia um tempo de operação de apenas 5 a 8 minutos, inviabilizando o uso prolongado em locais sem alimentação elétrica, especialmente em operações discretas. Para contornar essa limitação, foi utilizada uma bateria externa (\emph{power bank}) de aproximadamente 4000\,mAh, o que aumentou significativamente a autonomia do sistema e possibilitou a execução contínua do experimento por períodos estendidos.

    Para garantir a privacidade dos participantes e seguir princípios éticos de pesquisa, as credenciais capturadas passaram por um processo de anonimização antes de serem armazenadas. O procedimento completo de registro e anonimização das tentativas de login está descrito no Algoritmo~\ref{alg:anonimizacao}.

    \begin{algorithm}[H]
    \caption{Anonimização e Registro de Tentativas de Login}
    \label{alg:anonimizacao}
    \begin{algorithmic}[1]
        \Require \textit{email}, \textit{senha}
        \Ensure Dados anonimizados registrados em arquivo JSON
        
        \State \textit{domínio} $\gets$ parte após `@` em \textit{email}
        \State \textit{emailAnon} $\gets$ primeiros 2 caracteres de \textit{email} $+ \text{``****@''} +$ \textit{domínio}
        \State \textit{senhaTam} $\gets$ comprimento de \textit{senha}
        \State \textit{timestamp} $\gets$ data e hora atual
        
        \State \textit{dados} $\gets$ \{email: \textit{emailAnon}, senha\_tamanho: \textit{senhaTam}, timestamp: \textit{timestamp}\}
        
        \If{arquivo \texttt{credenciais\_anonimas.json} existe}
            \State \textit{logs} $\gets$ conteúdo do arquivo em JSON
        \Else
            \State \textit{logs} $\gets$ lista vazia
        \EndIf
        
        \State Adicionar \textit{dados} à lista \textit{logs}
        \State Salvar \textit{logs} no arquivo \texttt{credenciais\_anonimas.json} em formato JSON
        
        \State Marcar erro de login na sessão
        \State Redirecionar usuário para \texttt{index.php}
    \end{algorithmic}
\end{algorithm}

    O algoritmo inicia com a extração do domínio do e-mail informado e a anonimização parcial do endereço, mantendo apenas os dois primeiros caracteres seguidos por asteriscos e o domínio original. Em seguida, registra-se o tamanho da senha (sem armazenar seu conteúdo) e um \emph{timestamp} com data e hora da tentativa. Esses dados são organizados em um objeto JSON e inseridos em um arquivo local, preservando tentativas anteriores. Caso o arquivo já exista, ele é carregado e atualizado com a nova entrada. Por fim, a tentativa é marcada como falha na sessão e o usuário é redirecionado à página inicial, mantendo a simulação de erro de login e evitando suspeitas por parte da vítima.
