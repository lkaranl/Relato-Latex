\documentclass[12pt,a4paper]{article}

% --- Pacotes Básicos ---
\usepackage[utf8]{inputenc}
\usepackage[T1]{fontenc}
\usepackage[brazilian]{babel}
\usepackage{helvet} % Fonte similar à Arial
\renewcommand{\familydefault}{\sfdefault}
\usepackage{indentfirst}
\usepackage{setspace}
\usepackage{geometry}
\usepackage{titlesec}
\usepackage{caption}
\usepackage{etoolbox}

% --- Configuração de Margens  ---
\geometry{
    a4paper,
    top=3.5cm,
    left=3.0cm,
    right=2.5cm,
    bottom=2.5cm
}

% --- Espaçamento entre linhas [cite: 34] ---
\onehalfspacing % 1,5 para o corpo do texto

% --- Configuração de Seções [cite: 37, 38, 39] ---
\titleformat{\section}
  {\normalfont\normalsize\bfseries}{\thesection}{1em}{}
\titlespacing*{\section}{0pt}{1.5ex}{6pt}

% --- Títulos sem numeração (Resumo, Referências) [cite: 41] ---
\titleformat{name=\section,numberless}
  {\normalfont\normalsize\bfseries}{}{0pt}{}

% --- Comandos para Citações Longas (Tamanho 10, Espaço Simples)  ---
\newenvironment{citacaolonga}
  {\begin{list}{}{
    \small % Tamanho 10 aproximado
    \singlespacing
    \leftmargin=4cm
    \rightmargin=0pt}
    \item\relax}
  {\end{list}}

% --- Início do Documento ---
\begin{document}

% --- Título [cite: 7] ---
\begin{center}
    \singlespacing
    \textbf{\fontsize{12}{14}\selectfont TÍTULO EM MAIÚSCULAS, NEGRITO E CENTRALIZADO (MÁXIMO 2 LINHAS)}
\end{center}

\vspace{0.5cm}

% --- Categoria e Área Temática [cite: 8] ---
\noindent \textbf{CATEGORIA:} RELATO DE EXPERIÊNCIA \\
\textbf{ÁREA TEMÁTICA:} Inserir Área Aqui

\vspace{0.5cm}

% --- Autores [cite: 9, 10, 18] ---
\noindent {\fontsize{11}{13}\selectfont Nome Completo do Primeiro Autor\footnote{Formação. Local de atuação (IES). Email de contato.}} \\
\noindent {\fontsize{11}{13}\selectfont Nome Completo do Segundo Autor\footnote{Afiliação e contato.}} \\
\noindent {\fontsize{11}{13}\selectfont Nome Completo do Terceiro Autor\footnote{Afiliação e contato.}}

\vspace{1cm}

% --- Resumo [cite: 11, 12, 14] ---
\section*{Resumo}
\begin{spacing}{1.0}
{\fontsize{12}{14}\selectfont 
O resumo deve conter entre 150 e 250 palavras em parágrafo único. Deve apresentar o objetivo, o método e os resultados ou contribuições de forma concisa, escrito na terceira pessoa.
}
\end{spacing}

\vspace{0.3cm}
\noindent \textbf{Palavras-Chave:} Palavra 1. Palavra 2. Palavra 3. [cite: 13]

\vspace{1cm}

% --- Seções do Relato [cite: 42, 43] ---
\section{INTRODUÇÃO}
A introdução deve contextualizar o tema, a ideia inicial, o referencial teórico breve, objetivos e a justificativa[cite: 43]. 

\section{MATERIAIS E MÉTODOS / PROCEDIMENTOS METODOLÓGICOS}
Descrever de forma sucinta o método, universo, população, técnicas de coleta e análise de dados[cite: 51, 52].

\section{RESULTADOS E DISCUSSÃO}
Descrever o que foi encontrado na experiência, articulando com o referencial teórico[cite: 83, 84].

% Exemplo de Tabela/Figura [cite: 86, 95]
\begin{center}
    \captionof{table}{Título da Tabela na parte superior}
    \small % Notas/Fontes em tamanho 10 [cite: 96]
    \noindent Fonte: Os autores (2026). [cite: 98]
\end{center}

\section{CONSIDERAÇÕES FINAIS OU CONCLUSÃO}
Reflexões resumidas, sem novas citações, destacando contribuições e recomendações[cite: 114, 115].

\newpage

% --- Referências [ABNT NBR 6023:2018] ---
\section*{REFERÊNCIAS}
\begin{spacing}{1.0}
\fontsize{11}{13}\selectfont
\raggedright % Alinhamento à esquerda
\setlength{\parindent}{0pt} % Sem recuo de parágrafo
\setlength{\parskip}{\baselineskip} % Um espaço simples em branco entre referências

% REGRA: Lista única (não separar sites). Autores repetidos devem ter o nome repetido (não usar _______).

SOBRENOME, Nome. \textbf{Título do livro}. Local: Editora, ano.

SOBRENOME, Nome. Título do artigo. \textbf{Título do Periódico}, Local, v. X, n. Y, p. 00-00, ano.

AUTOR, Nome. \textbf{Título da obra online}. Disponível em: <http://www.site.com.br>. Acesso em: dia mês. ano.
\end{spacing}

\end{document}